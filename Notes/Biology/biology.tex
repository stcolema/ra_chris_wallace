\documentclass[a4paper,10pt]{article}

\usepackage[utf8]{inputenc}

%%% PAGE DIMENSIONS
\usepackage{geometry} % to change the page dimensions
\geometry{a4paper} % or letterpaper (US) or a5paper or....
%\geometry{margin=2in} % for example, change the margins to 2 inches all round
%\geometry{left=2.75cm, right=2.75cm, top=3.5cm, bottom=4.5cm}

\usepackage{graphicx} % support the \includegraphics command and options

%%% BibTex packages (url for website references)
\usepackage[english]{babel}
\usepackage[numbers]{natbib}
% \usepackage{url}
% \usepackage{Biblatex}

%For inclusion of hyperlinks
\usepackage{hyperref}
\hypersetup{
    colorlinks=true,
    linkcolor=blue,
    filecolor=magenta,      
    urlcolor=cyan,
}

%BibTex stuff and referencing sections by name 
\urlstyle{same}
\usepackage{nameref} 


% \usepackage{natbib}

%opening
\title{Biology notes}
\author{Stephen D. C.}

\begin{document}

\maketitle

\begin{abstract}
Some comments on relevant biology.

\end{abstract}

\section{Expression quantitative trait loci}
The last two decades have seen a huge body of research focused on genome variability due to its 
relevance in the risk of disease experienced by individuals. Fundamental to this study is 
understanding the effect different genome variants have; i.e. understanding how this change 
in genome translates to a different phenotype. This means we must investigate the change a 
variant effects within the cell. Ideally this information allows biological insight into the 
aetiology and nature of disease or of the phenotype. Genome-wide association studies (GWAS) 
\cite{manolio2010genomewide} have shown that the majority of these variants are located within 
the non-coding regions of the genome implying that they are involved in gene regulation. These 
sites that explain some of the phenotypic variance are referred to as expression quantitative 
trait loci (eQTL).

eQTLs have transformated the study of genetics. They provide a comprehensuible, accessible and 
most importantly interpretable loecular link between genetic variation and phenotype. Standard 
eQTL analysis involves a direction association test between markers of genetic variation, 
typically using data collected from tens to hundreds of people.

This analysis can be proximal or distal.
\begin{itemize}
 \item Proximal: immedietely responsible for causing some observed result;
 \item Distal: (also \emph{ultimate}) higher-level than proximal. The true cause for an event or 
 result.
\end{itemize}
Consider the example of a ship sinking. This could have a \emph{proximate} cause such as the ship 
being holed beneath the waterline leading to water entering the ship; this resulted in the ship 
becoming denser than water and it sank. However, the \emph{distal} cause could be the ship hit a 
rock tearing open the hull leading to the sinking.

In terms of eQTLs, we designate proximal effects as \emph{cis-eQTL} and distal causes as 
\emph{trans-eQTL}. We normally consider an eQTL to be cis-regulating if it is within 1MB of the 
gene transcription start site (TSS) and trans-regulating if it is more than 5MB upstream or 
downstream of the TSS or if found on a different chromosome.

trans-eQTL are hard to find. They have weaker effects than cis-eQTL and thus require greater power 
in the experiment \cite{dixon2007genome}. For some context, \citet{burgess2010principles} claims 
that 449 donors provide low power in terms of finding trans-eQTL. As the power of experiments increases 
more trans-eQTL are observed and cis-eQTL are shown to be generally tissue agnostic 
\cite{gtex2017genetic}. Preivous results suggested cis-eQTL would be have tissue-specific effects, 
but the increase in experimental power revealed that this is not the case \cite{grundberg2012mapping}. 
The current power present in many genetic experiments is enough to observe trans-eQTLs and indicates 
these have tissue-specific properties \cite{grundberg2012mapping} \cite{gtex2017genetic}. It is 
possible that this result might be shown to be an artifact of insufficient power. However, for now 
we assume it is true and that trans-eQTL are are more likely to display tissue-specifc behaviour 
than cis-eQTL.

\citet{nica2013expression} reccommend investigating groups of cis-eQTL affecting a gene network that 
when perturbed results in a disease state. They claim this is far higher powered than the classical 
approach. This claim is supported by the findings of \citet{vosa2018unraveling} who found that 
associations between \emph{polygenic risk scores} and gene expression (this assoication is referred 
to as ``expression quantitative trait score'' (eQTS) in \cite{vosa2018unraveling}) contained the 
most biological information about disease in a comparison of cis-eQTL, trans-EQTL and eQTS.

\bibliographystyle{plainnat}
\bibliography{biology}

\end{document}
