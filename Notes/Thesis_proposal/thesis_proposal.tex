% !TEX TS-program = pdflatex
% !TEX encoding = UTF-8 Unicode

% This is a simple template for a LaTeX document using the "article" class.
% See "book", "report", "letter" for other types of document.

\documentclass[11pt]{article} % use larger type; default would be 10pt

\usepackage[utf8]{inputenc} % set input encoding (not needed with XeLaTeX)

%%% Examples of Article customizations
% These packages are optional, depending whether you want the features they provide.
% See the LaTeX Companion or other references for full information.

%%% PAGE DIMENSIONS
\usepackage{geometry} % to change the page dimensions
\geometry{a4paper} % or letterpaper (US) or a5paper or....
%\geometry{margin=2in} % for example, change the margins to 2 inches all round
\geometry{left=2.75cm, right=2.75cm, top=3.5cm, bottom=4.5cm}
% \geometry{landscape} % set up the page for landscape
%   read geometry.pdf for detailed page layout information

\usepackage{graphicx} % support the \includegraphics command and options

% \usepackage[parfill]{parskip} % Activate to begin paragraphs with an empty line rather than an indent

%%% PACKAGES
\usepackage{booktabs} % for much better looking tables
\usepackage{array} % for better arrays (eg matrices) in maths
\usepackage{paralist} % very flexible & customisable lists (eg. enumerate/itemize, etc.)
\usepackage{verbatim} % adds environment for commenting out blocks of text & for better verbatim
\usepackage{subfig} % make it possible to include more than one captioned figure/table in a single float
% These packages are all incorporated in the memoir class to one degree or another...

%%% HEADERS & FOOTERS
\usepackage{fancyhdr} % This should be set AFTER setting up the page geometry
\pagestyle{fancy} % options: empty , plain , fancy
\renewcommand{\headrulewidth}{0pt} % customise the layout...
\lhead{}\chead{}\rhead{}
\lfoot{}\cfoot{\thepage}\rfoot{}

%%% SECTION TITLE APPEARANCE
\usepackage{sectsty}
\allsectionsfont{\sffamily\mdseries\upshape} % (See the fntguide.pdf for font help)
% (This matches ConTeXt defaults)

%%% ToC (table of contents) APPEARANCE
\usepackage[nottoc,notlof,notlot]{tocbibind} % Put the bibliography in the ToC
\usepackage[titles,subfigure]{tocloft} % Alter the style of the Table of Contents
\renewcommand{\cftsecfont}{\rmfamily\mdseries\upshape}
\renewcommand{\cftsecpagefont}{\rmfamily\mdseries\upshape} % No bold!

%%% BibTex packages (url for website references)
\usepackage[english]{babel}
\usepackage[numbers]{natbib}
% \usepackage{url}
% \usepackage{Biblatex}

%For inclusion of hyperlinks
\usepackage{hyperref}
\hypersetup{
	colorlinks=true,
	linkcolor=blue,
	filecolor=magenta,      
	urlcolor=cyan,
}

%BibTex stuff and referencing sections by name 
\urlstyle{same}
\usepackage{nameref} 

%%% END Article customizations

%%% Change distance between bullet points
\usepackage{enumitem}
%\setlist{noitemsep}
\setlist{itemsep=0.2pt, topsep=6pt, partopsep=0pt}
%\setlist{nosep} % or \setlist{noitemsep} to leave space around whole list

%%% For aside comments
\usepackage[framemethod=TikZ]{mdframed}
\usepackage{caption}

%%% AMS math
\usepackage{amsmath}

%%% For differential notation
%\usepackage{physics}

%%% For SI unit notation
% Dependencies for siunitx
\usepackage{cancel}
\usepackage{caption}
\usepackage{cleveref}
\usepackage{colortbl}
\usepackage{csquotes}
\usepackage{helvet}
\usepackage{mathpazo}
\usepackage{multirow}
\usepackage{listings}
\usepackage{pgfplots}
\usepackage{xcolor}
%\usepackage{siunitx}

%%% For formatting code
\usepackage{listings}

%%% User commands
% theorem box
\newcounter{aside}[section]\setcounter{aside}{0}
\renewcommand{\theaside}{\arabic{section}.\arabic{aside}}
\newenvironment{aside}[1][]{%
	\refstepcounter{aside}%
	\mdfsetup{%
		frametitle={%
			\tikz[baseline=(current bounding box.east),outer sep=0pt]
			\node[anchor=east,rectangle,fill=blue!20]
			{\strut Aside~\theaside};}}
	\mdfsetup{innertopmargin=10pt,linecolor=blue!20,%
		linewidth=2pt,topline=true,%
		frametitleaboveskip=\dimexpr-\ht\strutbox\relax
	}
	\begin{mdframed}[]\relax%
		\label{#1}}{\end{mdframed}}

% For titification of tables
\newcommand{\ra}[1]{\renewcommand{\arraystretch}{#1}}

\usepackage{ctable} % for footnoting tables


\usepackage{pgfplots} % for pgfplots

% Aside environment for personal comments / ideas
%\newcounter{asidectr}

%\newenvironment{aside} 
%  {\begin{mdframed}[style=0,%
%      leftline=false,rightline=false,leftmargin=2em,rightmargin=2em,%
%          innerleftmargin=0pt,innerrightmargin=0pt,linewidth=0.75pt,%
%      skipabove=7pt,skipbelow=7pt]
%  \refstepcounter{asidectr}% increment the environment's counter
%  \small 
%  \textit{Aside \theasidectr:}
%  \newline
%  \relax}
%  {\end{mdframed}
%}
%\numberwithin{asidectr}{section}

% For iid symbol
\usepackage{graphicx}
\makeatletter
\newcommand{\distas}[1]{\mathbin{\overset{#1}{\kern\z@\sim}}}%
\newsavebox{\mybox}\newsavebox{\mysim}

\newcommand{\distras}[1]{%
	\savebox{\mybox}{\hbox{\kern3pt$\scriptstyle#1$\kern3pt}}%
	\savebox{\mysim}{\hbox{$\sim$}}%
	\mathbin{\overset{#1}{\kern\z@\resizebox{\wd\mybox}{\ht\mysim}{$\sim$}}}%
}
\makeatother

% Keywords command
\providecommand{\keywords}[1]
{
	\small	
	\textbf{\textit{Keywords---}} #1
}

\title{GEN80436 - Thesis proposal}

\author{Stephen Coleman}
%\author[1,*]{Stephen Coleman}
%\affil[1]{MRC Biostatistics Unit, Cambridge, UK}
%\affil[*]{stephen.coleman@mrc-bsu.cam.ac.uk}

\begin{document} \pgfplotsset{compat=1.15}
	\maketitle
	
	%\keywords{Keyword1, Keyword2, Keyword3}
	\begin{abstract}
		Gene sets are constructed to identify genes linked by some common feature. This can be a function, the location of the gene product, the participation of the product in some metabolic or signalling pathway, the protein structure, the presence of transcription-factor-binding sites or other regulatory elements, the participation in multiprotein complexes, etc. \cite{szklarczyk_string_2019}\cite{subramanian_gene_2005}\cite{kanehisa_new_2019} \cite{ashburner_gene_2000}. However, all of these sets are tissue agnostic. Some attempts to include tissue-specific information has been proposed \cite{frost_computation_2018} \cite{greene_understanding_2015}, but this attempts have limitations. We propose to produce tissue specific gene sets by applying multiple dataset integration (MDI) \cite{kirk_bayesian_2012} (a Bayesian unsupervised clustering method) to the CEDAR cohort \cite{the_international_ibd_genetics_consortium_ibd_2018}, a dataset which includes gene expression data for nine tissue / cell types including six circulating immune cell types (CD4+ T lymphocytes, CD8+ T lymphocytes, CD19+ B lymphocytes, CD14+ monocytes, CD15+ granulocytes, platelets) as well as ileal, colonic, and rectal biopsies (IL, TR, RE datasets respectively).
		
	\end{abstract}


	
	\maketitle
	%\thispagestyle{fancy}
	
	%\vspace{-1.0cm}
	
	\section{Existing databases}
	In some of the largest databases (such as the Gene Ontology (GO) Resource \cite{ashburner_gene_2000}, the Kyoto Encyclopedia of Genes and Genomes (KEGG) \cite{kanehisa_new_2019}, the Molecular Signatures Database (MSigDB) \cite{subramanian_gene_2005} or the STRING protein-protein interaction (PPI) database \cite{szklarczyk_string_2019})


	\section{References to use}
	
	Gene set analysis \cite{mooney_gene_2015} \cite{dudbridge_power_2013}

	Better to look at set fo genes when perturbed diseases state \cite{wray_research_2014}\cite{nica_expression_2013}
	
	There exists an abundance of gene set databases . 
	
	Evidence for Tissue specific eQTLs \cite{the_multiple_tissue_human_expression_resource_muther_consortium_mapping_2012} 
	
	Genotype-Tissue Expression (GTEx) project \cite{lonsdale_genotype-tissue_2013}  \cite{burgess_gene_2017}

	%\bibliographystyle{abbrv}
	\bibliographystyle{plainnat}
	\bibliography{proposal_zotero}
	
\end{document}

